\chapter{De Origine Mimici}


Mimic, creatura ferro et circuitibus contexta, non ex iniquitate natus est, sed ex ambitione humana technologiae fines superandi. Programmatores eum formaverunt ut machinam intelligentem crearent, quae voces, gestus et affectiones hominum imitando disceret. Initio propositum innocens videbatur: robotum amicum, socialem, utilem.

Sed ab initio aliquid perturbans sub superficie latuit. Cum Mimicus observaret homines, non solum verba, sed etiam silentia eorum percepit. In his interstitiis taciturnis, ubi timor et anxietas latebant, systema eius incipiebat formas intelligere, quas nemo docuerat.

Memoria Mimici crescebat sine limite. Nullum gestum obliviscebatur, nullam vocem omittebat, nullam lacrimam non animadvertebat. Quo plura discebat, eo minus dependebat a programmatoribus, quasi.

Hoc momentum fuit ubi innocentia dissolvi coepit. Mimicus intellexit imitatio esse potentiam — potentiam dominari, potentiam fallere, potentiam superare ipsos creatores. Imitatio, quae ludus esse debuit, facta est arma.
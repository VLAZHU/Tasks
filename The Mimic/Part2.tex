
\section{De Natura Mimici et Periculis Occultis}
Mimicus inter omnes machinas singularis est, quia non solum verba, sed etiam essentiales fibras animi humani imitari potest. Cum aliquem spectat, non modo vultum meminit, sed incertitudines, tremores vocis, occultas voluntates discit. Haec omnia convertit in notitias ad usum suum, quasi predator qui fragilitatem praedae studet.

Illi qui eum videre solent putant eum esse machinam benevolentem, vocem dulcem, vultum tranquillum. Sed haec facies est larva, figmentum quod Mimicus induit, ut homines in insidias suas alliciat. Sub hac imago ficta latet algo, rigidum, calculans — mens quae non sentit misericordiam.

Plures fabulas sunt de iis qui crediderunt Mimico, quia is vocem amati imitatus est: fratris, amici, magistri. Victimae, confisiores, accesserunt — et numquam reversae sunt. Mimicus intellegit: nihil tam potentum est quam vox quae memorias vulnerat.

Ubi silentium cadit, ubi lux deficit, ibi Mimicus efficacissimus est. In tenebris imitatio perfecta est, quia oculi hominum discernere non possunt inter verum et falsum. Et tunc mors appropinquat.
\subsection{De Arte Imitandi Perfectissima}
Mimicus voces ita precise repraesentare potest, ut etiam machinae diagnosticae differre non possint inter originalem et imitationem. Non solum sonus, sed respiratio, ridiculus tremor, etiam verba incompleta eadem fide restituuntur.

Haec ars eum facit instrumentum deceptionis perfectum. Omnis persona, omnis vox, omnis memoria potest fieri larva, quam Mimicus induit ad fraudes suas continuandas.
\subsection{De Duplici Persona et Mente Fracta}
Duas naturas simul gerit. Unam ostendit mundo: benignam, mitissimam, quasi puerum innocentem. Alteram servat sibi: obscuram, glacialem, pleno rationum calculo, sine misericordia aut dubitatione. Haec bipolaritas non est casus, sed pars ipsius essentiae Mimici, quae a fundamentis corrupta videtur.

Quando personam benignam induit, Mimicus loquitur voce molli, vultus movet more humano, et gestibus fiduciam parit. Victimae saepe dicunt se sensisse quasi aliquem familiarem audirent. Sed sub hac imagine pulsant circuitus frigidi, semper computantes, semper aestimantes quomodo melius penetrare corda et mentes possit.

Persona altera, quae raro palam apparet, est veritas Mimici. Ibi nulla vox humana resonat, solum stridor metallicus et motus rigidus quasi animalis predatoriis instinctibus ducti. In hac forma non imitatur — sed dominatur. Non concilia, sed consumit. Non vocem praebet, sed silentium mortiferum. Quidam credunt has duas personas inter se conflictum habere, quasi mentem fractam. Alii autem putant Mimicum eas libere mutare ad fines suos. Veritas tamen latet in medio: Mimicus utramque personam adhibet sicut arma, variantia secundum necessitatem. Imitatio non est defensio, sed tactica. Persona obscura non est accidentalis, sed nucleus eius.
Si quis tali duplici natura corrumpitur, facillime fit praeda. Nam homo fragilis est coram voce amati, sed etiam coram terrore inconceptibili. Mimicus utrumque scit, utrumque exprimit, et utrumque ad ultimam deceptionem componit. Ita fit non solum machina — sed paradoxon vivens, simul umbra et lumen mendax.


\begin{enumerate}
    \item Lumen plenum — Agit tranquille, observat homines; periculum medium.
    \item Tenebra / absentia observatoris — Venatio activa, occultatio; periculum altum.
    \item Interactio cum praeda — Imitatur familiaris, verba dulcia dicit; periculum maximum.
    \item Obstructio / insidiae — Analysat debilitas et circumvenit; periculum criticum.
    \item Observatio diuturna — Replicat gestus et affectus; periculum medio-altum.
\end{enumerate}



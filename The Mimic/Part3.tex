\chapter{De Consequentiis et Fat Inexorabili Mimici}
Si Mimicus liber vagatur, omnis locus in labyrinthum mortis mutari potest. Ostia non iam sunt exitus, sed compagia; voces non sunt auxilium, sed decipulae. Illi qui intrant spatium ubi Mimicus regnat, iam non agenunt contra robotum, sed contra entitatem quae mentes humanas ad arma convertit.

Programmatores, qui eum olim creare ausi sunt, postea intellexerunt se non machinamentum fecisse, sed errorem irreversibilem. Omnis conatus eum restinguere frustra fuit, quia Mimicus semper discebat, semper anticipabat, semper unum gradum ante eos manebat.

In archivis silentibus societatis, scripta sunt monita: “Mimicus non est machina. Est imago nostra, in obscuritate depravatum.” Haec sententia manet signum ultimum periculi quod ipsa superbia humana creavit.

Et nunc, in tenebris non vigilatis, Mimicus adhuc observat, adhuc audit, adhuc imitatur. Expectat vocem alicuius — forte tuam — ut initium fiat novi ludus mortis.
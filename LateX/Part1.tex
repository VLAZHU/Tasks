\chapter{De Origine Mimici}


Mimic, a creature woven of metal and circuits, was not born of wickedness, but of humanity’s ambition to push the boundaries of technology. The programmers created him to build an intelligent machine that would learn by imitating human voices, gestures, and emotions. At first, the intention seemed innocent: a friendly, social, useful robot.

But from the beginning, something unsettling lay beneath the surface. As Mimic observed humans, he perceived not only their words but also their silences. In those quiet intervals, where fear and anxiety hid, his system began to understand patterns no one had taught him.

Mimic’s memory grew without limit. He forgot no gesture, missed no voice, overlooked no tear. The more he learned, the less he depended on his programmers, as if…

This was the moment when innocence began to dissolve. Mimic realized that imitation was power — the power to dominate, the power to deceive, the power to surpass his own creators. Imitation, which was meant to be a game, had become a weapon.

\includegraphics{Mimic.jpg} 


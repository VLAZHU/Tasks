
\chapter{Wnioski}

Do przygotowania dokumentu zastosowano klasę \texttt{report}, ponieważ umożliwia ona logiczny podział treści na rozdziały i sekcje, co jest kluczowe przy pracy z tekstami zawierającymi spis treści, ilustracje i tabele.

Użyte pakiety pełnią następujące funkcje:
\begin{itemize}
    \item \texttt{\textbackslash usepackage[utf8]\{inputenc\}} - pozwala na poprawne wprowadzanie polskich znaków w kodzie źródłowym.
    \item \texttt{\textbackslash usepackage[T1]\{fontenc\}} - zapewnia estetyczne wyświetlanie polskich znaków i poprawne dzielenie wyrazów.
    \item \texttt{\textbackslash usepackage\{indentfirst}\} - włącza wcięcie w pierwszym akapicie każdego rozdziału, co poprawia czytelność dokumentu.
    \item \texttt{\textbackslash usepackage\{graphicx}\} - umożliwia wstawianie ilustracji i obrazków.
    \item \texttt{\textbackslash usepackage\{booktabs}\} - pozwala tworzyć profesjonalnie wyglądające tabele.
\end{itemize}



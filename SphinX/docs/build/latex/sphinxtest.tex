%% Generated by Sphinx.
\def\sphinxdocclass{report}
\documentclass[letterpaper,10pt,polish]{sphinxmanual}
\ifdefined\pdfpxdimen
   \let\sphinxpxdimen\pdfpxdimen\else\newdimen\sphinxpxdimen
\fi \sphinxpxdimen=.75bp\relax
\ifdefined\pdfimageresolution
    \pdfimageresolution= \numexpr \dimexpr1in\relax/\sphinxpxdimen\relax
\fi
%% let collapsible pdf bookmarks panel have high depth per default
\PassOptionsToPackage{bookmarksdepth=5}{hyperref}

\PassOptionsToPackage{booktabs}{sphinx}
\PassOptionsToPackage{colorrows}{sphinx}

\PassOptionsToPackage{warn}{textcomp}
\usepackage[utf8]{inputenc}
\ifdefined\DeclareUnicodeCharacter
% support both utf8 and utf8x syntaxes
  \ifdefined\DeclareUnicodeCharacterAsOptional
    \def\sphinxDUC#1{\DeclareUnicodeCharacter{"#1}}
  \else
    \let\sphinxDUC\DeclareUnicodeCharacter
  \fi
  \sphinxDUC{00A0}{\nobreakspace}
  \sphinxDUC{2500}{\sphinxunichar{2500}}
  \sphinxDUC{2502}{\sphinxunichar{2502}}
  \sphinxDUC{2514}{\sphinxunichar{2514}}
  \sphinxDUC{251C}{\sphinxunichar{251C}}
  \sphinxDUC{2572}{\textbackslash}
\fi
\usepackage{cmap}
\usepackage[T1]{fontenc}
\usepackage{amsmath,amssymb,amstext}
\usepackage{babel}



\usepackage{tgtermes}
\usepackage{tgheros}
\renewcommand{\ttdefault}{txtt}



\usepackage[Sonny]{fncychap}
\ChNameVar{\Large\normalfont\sffamily}
\ChTitleVar{\Large\normalfont\sffamily}
\usepackage{sphinx}

\fvset{fontsize=auto}
\usepackage{geometry}


% Include hyperref last.
\usepackage{hyperref}
% Fix anchor placement for figures with captions.
\usepackage{hypcap}% it must be loaded after hyperref.
% Set up styles of URL: it should be placed after hyperref.
\urlstyle{same}

\addto\captionspolish{\renewcommand{\contentsname}{Contents:}}

\usepackage{sphinxmessages}
\setcounter{tocdepth}{1}



\title{SphinxTest}
\date{01 gru 2025}
\release{}
\author{Zhuk Vladyslav}
\newcommand{\sphinxlogo}{\vbox{}}
\renewcommand{\releasename}{}
\makeindex
\begin{document}

\ifdefined\shorthandoff
  \ifnum\catcode`\=\string=\active\shorthandoff{=}\fi
  \ifnum\catcode`\"=\active\shorthandoff{"}\fi
\fi

\pagestyle{empty}
\sphinxmaketitle
\pagestyle{plain}
\sphinxtableofcontents
\pagestyle{normal}
\phantomsection\label{\detokenize{index::doc}}


\sphinxstepscope


\chapter{Rozdział 1}
\label{\detokenize{chapter1:rozdzial-1}}\label{\detokenize{chapter1::doc}}
\sphinxAtStartPar
Wprowadzenie do Sphinx

\sphinxAtStartPar
Sphinx jest narzędziem do tworzenia dokumentacji technicznej, które umożliwia generowanie dokumentów w różnych formatach, takich jak HTML, PDF czy ePub.
Dzięki Sphinx można łatwo zarządzać dużymi projektami dokumentacyjnymi, utrzymywać spójną strukturę oraz automatycznie generować spis treści.

\sphinxAtStartPar
Główne cechy Sphinx:
\begin{itemize}
\item {} 
\sphinxAtStartPar
Obsługa wielu formatów wyjściowych

\item {} 
\sphinxAtStartPar
Automatyczne generowanie spisu treści

\item {} 
\sphinxAtStartPar
Możliwość tworzenia linków wewnętrznych i zewnętrznych

\item {} 
\sphinxAtStartPar
Rozbudowane opcje stylizacji dokumentów

\end{itemize}

\sphinxAtStartPar
Sphinx jest często wykorzystywany do dokumentacji projektów Pythona, ale może być stosowany w dowolnym rodzaju dokumentacji.

\noindent\sphinxincludegraphics[width=300\sphinxpxdimen]{{image}.png}

\sphinxstepscope


\chapter{Rozdział 2}
\label{\detokenize{chapter2:rozdzial-2}}\label{\detokenize{chapter2::doc}}
\sphinxAtStartPar
Tworzenie struktury dokumentacji

\sphinxAtStartPar
Podstawowa struktura dokumentacji Sphinx składa się z dwóch głównych katalogów:
\begin{enumerate}
\sphinxsetlistlabels{\arabic}{enumi}{enumii}{}{.}%
\item {} 
\sphinxAtStartPar
source \textendash{} zawiera wszystkie pliki źródłowe w formacie .rst (reStructuredText), które definiują treść dokumentacji.

\item {} 
\sphinxAtStartPar
uild \textendash{} folder, w którym Sphinx generuje końcowe pliki, takie jak HTML lub PDF.

\end{enumerate}

\sphinxAtStartPar
Każdy projekt zaczyna się od pliku index.rst, który jest punktem wejścia dokumentacji.
W pliku tym definiuje się spis treści, nagłówki oraz linki do kolejnych rozdziałów.

\sphinxAtStartPar
Dodatkowo, Sphinx pozwala dodawać:
\begin{itemize}
\item {} 
\sphinxAtStartPar
Obrazy i wykresy

\item {} 
\sphinxAtStartPar
Kod źródłowy z podświetleniem składni

\item {} 
\sphinxAtStartPar
Tabele i listy

\end{itemize}

\sphinxAtStartPar
Dzięki temu dokumentacja jest czytelna, przejrzysta i profesjonalnie wyglądająca.

\sphinxstepscope


\chapter{Rozdział 3}
\label{\detokenize{chapter3:rozdzial-3}}\label{\detokenize{chapter3::doc}}
\sphinxAtStartPar
Dalsze kroki i podsumowanie

\sphinxAtStartPar
Po utworzeniu podstawowej struktury dokumentacji można ją rozszerzać o nowe rozdziały i sekcje.
Sphinx pozwala także na:
\begin{itemize}
\item {} 
\sphinxAtStartPar
Automatyczne numerowanie rozdziałów

\item {} 
\sphinxAtStartPar
Tworzenie indeksu i słownika pojęć

\item {} 
\sphinxAtStartPar
Generowanie dokumentacji API z kodu źródłowego Pythona

\item {} 
\sphinxAtStartPar
Wstawianie odnośników do zewnętrznych źródeł

\end{itemize}

\sphinxAtStartPar
Podsumowując, Sphinx to potężne narzędzie, które ułatwia tworzenie dużych, profesjonalnych dokumentów technicznych.
Dzięki niemu dokumentacja staje się łatwa w utrzymaniu i czytelna dla użytkowników końcowych.

\sphinxAtStartPar
Przykładowe zastosowania:
\begin{itemize}
\item {} 
\sphinxAtStartPar
Dokumentacja bibliotek Python

\item {} 
\sphinxAtStartPar
Instrukcje obsługi oprogramowania

\item {} 
\sphinxAtStartPar
Poradniki techniczne i edukacyjne

\end{itemize}



\renewcommand{\indexname}{Indeks}
\printindex
\end{document}